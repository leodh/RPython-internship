\documentclass{beamer}

\include{pythonlisting}

\usepackage{beamerthemesplit} % Activate for custom appearance
\usetheme{Warsaw}



\newtheorem{question}{Question}

\title{RPython}
\author{Leonard de HARO}
\date{\today}

\begin{document}

\frame{\titlepage}

\section[Outline]{}
\frame{\tableofcontents}

\section{Introduction}
\frame
{
  \frametitle{The Pypy Project}

  \begin{itemize}
 	 \item<1-> New Python VM, written in RPython\\
  			\uncover<3->{\emph{The Interpreter}}
  	\item<2-> New language to design VMs: RPython\\
  			\uncover<3->{\emph{The Translation Toolchain}}
  \end{itemize}
  
\uncover<4->
{   
  \begin{fact}
  The Translation Toolchain provides a JIT compiler on demand!
  \end{fact}
 } 
}

\section{Just-In-Time compilation}

\frame
{
  \frametitle{Sure... But what's a JIT compiler?}
  
  Two kinds:
	\begin{itemize}
		\item Method JITs \\
			\textit{e.g.} HotSpot in JVM)
		\item Tracing JITs \\
			\textit{e.g.} Pypy
	\end{itemize}
}

\frame
{
  \frametitle{Method JITs}

	\begin{itemize}
	  	\item<1-> Works on the bytecode (linear)
	  	\item<2-> Notices "Hot Spots"
	  	\item<3-> Compiles them (native code)
	  	\item<4-> Uses the native code version
	\end{itemize}
  
}

\frame
{
  \frametitle{Tracing JIT}
  	
	\begin{itemize}
		\item<1-> Works during execution
		\item<2-> Finds a "hot loop"
		\item<3-> Traces the execution
		\item<4-> Optimizes it (including guards)
		\item<5-> Uses the optimized traced version
	\end{itemize}
\uncover<6->
{
	\begin{fact}
	Pypy is a \textbf{meta-tracing JIT} compiler: feed it a properly annotated interpreter, it gives you back a tracing JIT	 interpreter.
	\end{fact}
}
}

\frame
{
	\begin{question}
	Examples of Pypy's JIT all work with bytecode. Can we make it work on ASTs?
	\end{question}
}		


\section{RPython as a language}
\frame
{
  \frametitle{General properties of RPython}
  
  \begin{itemize}
  	\item<1-> Strict and valid subset of Python
	\item<2-> Statically typed (with exception)
	\item<3-> Output C (when used in TT)
	\item<4-> \textbf{Creates JITing VMs} (for under 10 lines of code)
	\item<5-> Still in development although useable
  \end{itemize}
}

\frame
{
  \frametitle{Writing a JIT VM}
  
  \begin{itemize}
  	\item<1-> Write your interpreter
	\item<2-> Add RPython instructions for translation
	\item<3-> Add RPython instructions for JITing
		\begin{itemize}
			\item<4-> \texttt{can\_enter\_jit}
			\item<5-> \texttt{jit\_merge\_point}
			\item<6-> Declare \textit{Red} and \textit{Green} variables
		\end{itemize}
	\item<7-> Optimize (\textit{e.g.} insert \textit{assert} to help the interpreter or  use fixed-size lists)
  \end{itemize}
}

\section{Demonstration}
\frame
{
\begin{center}
\textbf{Demo Time !}\\
(see Andrew Brown's tutorial on Pypy's Blog)
\end{center}
}


%\begin{frame}[fragile]{Some Python Code}
%\begin{python}
%def foo(x):
%	if x==0:
%		return 42
%	else:
%		return "42"
%\end{python}
%\end{frame}

\end{document}
